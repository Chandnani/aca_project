

\documentclass[10pt]{beamer}
\usepackage[utf8]{inputenc}
\usepackage[english]{babel}
\usepackage{amsthm}
\usepackage{amsmath}
\usepackage{amsfonts}
\usetheme{Madrid}
\usecolortheme{beaver}
\usefonttheme{structuresmallcapsserif}


%Information to be included in the title page:
\title{Eigenvalues and Eigenvectors}
\author{A.K.Lal}
\institute{IIT Kanpur}
\date{2017}
\begin{document}
\maketitle
\begin{frame}
\frametitle{Outline}
\tableofcontents
\end{frame}
\section{introduction}
\subsection{Characterstic Polynomial and Equation
}
 
\begin{frame}
\frametitle{Introduction}
\begin{definition}[Characteristic Polynomial, Characteristic Equation]
Let $A$ be a square matrix of order n. The polynomial $ \det (A - \lambda*I)$ is called the characteristic polynomial of $A$ and is denoted by $p_A(\lambda)$ (in short, $p(\lambda)$, if the matrix $A$ is clear from the context). The equation $p(\lambda) = 0$ is called the characteristic equation of $A$. If $\lambda \in \mathbb{F}$ is a solution of the characteristic equation $p(\lambda) = 0$, then $\lambda$ is called a characteristic value of $A$. 

\end{definition}
\begin{theorem}
Let $A \in \mathbb{M}_n(\mathbb{F})$. Suppose $\lambda = \lambda_0 \in \mathbb{F}$ is a root of the characteristic equation. Then
there exists a non-zero $v \in \mathbb{F}^n$ such that $Av = \lambda_0 v$.
\end{theorem}
\end{frame}
\subsection{Eigenvalues and Eigenvectors}
\begin{frame}
\frametitle{Definition of Eigenvalues and Eigenvectors}
\begin{definition}
Let $A \in \mathbb{M}_n (\mathbb{F})$ and let the linear system $Ax = \lambda x$ has a non-zero solution $x \in \mathbb{F}^n$ for some $\lambda \in \mathbb{F}$. Then
\begin{itemize}
\item $\lambda \in \mathbb{F}$ is called an eigenvalue of $A$,
\item $ x \in \mathbb{F}^n$ is called an eigenvector corresponding to the eigenvalue $\lambda$ of $A$, and
\item the tuple $(\lambda, x)$ is called an eigen-pair.
\end{itemize}
\end{definition}
\end{frame}
\section{Cayley Hamilton Theorem}
\begin{frame}
\frametitle{Cayley Hamilton Theorem}
\begin{theorem}
Let $A$ be a square matrix of order $n$. Then $A$
satisfies its characteristic equation. That is,
$A^n + a_{n-1} A^{n-1} + a_{n-2} A^{n-2} + ..... +  a_1A + a_0I = 0$
holds true as a matrix identity.
\end{theorem}
\end{frame}
\section{Diagonalization}
\subsection{Matrix Diagonalization}
\begin{frame}
\frametitle{Diagonalization}
\begin{definition}[Matrix Diagonalization]
A matrix $A$ is said to be diagonalizable if there exists
a non-singular matrix $P$ such that $P^{-1} AP$ is a diagonal matrix.
\end{definition}
\subsection{Condition for diagonalization}
\begin{theorem}[Condition for diagonalization]
Let $A \in \mathbb{M}_n(\mathbb{R})$. Then A is diagonalizable if and only if $A$ has $n$ linearly independent eigenvectors
\end{theorem}
\end{frame}
\end{document}

\grid
